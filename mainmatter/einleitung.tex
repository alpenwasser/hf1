% **************************************************************************** %
\chapter{Einleitung}
\label{chap:intro}
% **************************************************************************** %

Zur Bestimmung  von verschiedenen Eigenschaften physikalischer  Systeme werden
heute  \"ublicherweise Simulationstools  eingesetzt. Gen\"ugend Rechenleistung
vorausgesetzt,   lassen   sich   so   beinahe   beliebig   komplexe   Probleme
l\"osen. Vor der  Ankunft moderner  Rechner und der  zugeh\"origen m\"achtigen
Simulationssoftware   waren  Wissenschaftler   und   Ingenieure  aber   darauf
angewiesen,  ihre   komplexen  Probleme   auf  m\"oglichst  clevere   Art  und
Weise  zu  herunterzubrechen  und  zu vereinfachen,  um  entweder  analytische
N\"aherungsl\"osungen  zu  erhalten,  oder zumindest  numerische  Methoden  zu
entwickeln, deren L\"osungen ohne moderne Computer berechnet werden konnten.

Ein beliebtes Feld  solcher Ans\"atze war die  Bestimmung von Induktivit\"aten
verschiedener    Leiterkonfigurationen,   mit    vielen   Ver\"offentlichungen
verschiedener  Autoren  im  Verlaufe  der   Jahre. Im  Herbst  1907  wurde  im
\emph{Bulletin  of the  Bureau of  Standards} von  Edward B. Rosa  ein solcher
Artikel  ver\"offentlich,  in  dem  er  sich  mit  den  Selbstinduktivit\"aten
und    gegenseitigen   Induktivit\"aten    linearer   Leiter    besch\"aftigte
\cite{ref:inductance:rosa}.

In  dieser Arbeit  sollen  seine  Methoden mithilfe  moderner  Tools auf  ihre
Genauigkeit  \"uberpr\"uft werden. Dazu  wird  der Grossteil  des Artikels  in
Kapitel \ref{chap:rosa}  zuerst \"ubersetzt und ein  bisschen zusammengefasst,
um  anschliessend   in  Kapitel   \ref{chap:simu}  numerische   Beispiele  mit
Simulationsergebnissen zu vergleichen.

Die Entstehungsgeschichte dieser Arbeit gliedert sich in zwei Teile:

% ---------------------------------------------------------------------------- %
\subsection*{Phase 1 (Herbstsemester 2016)}
% ---------------------------------------------------------------------------- %
Um  das Verst\"andnis  f\"ur die  Materie zu  erarbeiten, wird  Rosa's Artikel
\"ubersetzt,  leicht  zusammengefasst  und  die  Grafiken  reproduziert. Dabei
werden  die meisten  Herleitungen ebenfalls  \"ubernommen, und  nicht nur  die
Endergebnisse  seiner \"Uberlegungen.   Dies  stellt sicher,  dass die  Arbeit
nicht  einfach ein  Copy-Paste-Exzess  von Endergebnissen  wird, sondern  dass
ich  mich  auch  wirklich  mit  der  Materie  auseinandersetze  und  versuche,
die  Gedankeng\"ange  nachzuvollziehen. Das   Eintippen  der  Gleichungen  ist
gl\"ucklicherweise ein  ausreichend langsamer  Prozess, dass das  Gehirn dabei
Zeit hat, sich seine Gedanken zum Getippten zu machen (statt einfach robotisch
Zeichenketten abzutippen). Das \"Ubersetzen des  Textes hilft ebenfalls dabei,
den Inhalt gedanklich zu verarbeiten.

Ich  betrachte  die  Arbeit  in  dieser  Phase  als  Arbeitsdokument. Folglich
sind  an verschiedenen  Orten Unklarheiten,  Unsauberkeiten oder  andere Dinge
notiert,  die  in der  zweiten  Phase  noch meiner  Aufmerksamkeit  bed\"urfen
(orange  Boxen). Ebenfalls  fehlen  nat\"urlich  noch  das  Abstract  und  die
Schlussfolgerungen, und das Layout bedarf noch einiger Optimierung.

Hauptergebnis dieses Teils ist das Kapitel \ref{chap:rosa}.

% ---------------------------------------------------------------------------- %
\subsection*{Phase 2 (Fr\"uhlingssemester 2017)}
% ---------------------------------------------------------------------------- %
Die analytischen L\"osungen werden  vom \emph{cgs}-System ins \emph{SI}-System
konvertiert,  mit  numerischen  Beispielen  erg\"anzt und  die  Resultate  mit
Simulationen verglichen.

Hauptergebnis dieses  Teils ist das Kapitel  \ref{chap:simu}. Ebenfalls werden
die  in  Phase  1  noch   nicht  geschriebenen  Abschnitte  verfasst  und  der
Feinschliff am Dokument angebgracht.
