% **************************************************************************** %
\chapter{Analytische L\"osungen}
\label{chap:rosa}
% **************************************************************************** %

% ---------------------------------------------------------------------------- %
\section{Selbstinduktivit\"at eines geraden, zylindrischen Drahtes}
\label{sec:selfStraightCylWire}
% ---------------------------------------------------------------------------- %

\schematic{selfStraightCylWire}

Die magnetische Feldst\"arke am Punkt $P$ aufgrund eines L\"angenelements $dy$
des Drahtes mit Strong $i$ durchflossenen Drahtes ist:

\begin{equation}
    \label{eq:sSCW:1}
    i \cdot \frac{dy}{c^2} \cdot \sin{\theta} 
    = 
    \frac{i \cdot a \cdot dy}{\left[ a^2 + (y - b)^2 \right]^{\frac{3}{2}}}
\end{equation}

\"Uber die  gesamte L\"ange  $l$ des  Drahtes und normiert  auf den  Strom $i$
integriert ergibt dies:

\begin{equation}
    \label{eq:sSCW:2}
    \frac{H}{i}
    = 
    \int_0^l 
    \frac{a \cdot dy}{\left[ a^2 + (y - b)^2 \right]^{\frac{3}{2}}}
    =
    \frac{l-b}{a \cdot \sqrt{a^2 - (l-b)^2}} + \frac{b}{a\cdot\sqrt{a^2+b^2}}
\end{equation}

Als n\"achstes  soll der  magnetische Fluss bestimmt  werden. Dieser berechnet
sich  \"uber das  Fl\"achenintegral der  magnetischen Flussdichte  $B$, welche
wiederum aus der magnetischen Feldst\"arke $H$ bestimmt wird gem\"ass:

\begin{equation}
    \label{eq:sSCW:3}
    B = H \cdot \mu
\end{equation}

Wobei  $\mu$   die  magnetische  Permeabilit\"at  des   Mediums  ist,  welches
vom  magnetischen  Feld durchflossen  wird. Somit  kann  ein Streifen  $d\Phi$
der  Breite  $dx$  des   magnetischen  Gesamtflusses  berechnet  werden  durch
Aufintegrieren des H-Feldes \"uber den Streifen der L\"ange $l$ und der Breite
$dx$:

\begin{equation}
    \label{eq:sSCW:4}
    \frac{d\Phi}{i} 
    = 
    \frac{dx \cdot \mu}{a} \cdot 
    \int_0^l 
    \left[ 
        \frac{l-b}{\sqrt{a^2 - (l-b)^2}} + \frac{b}{\sqrt{a^2+b^2}}
    \right]
    db
    =
    \frac{2\mu dx}{a}
    \left[
        \sqrt{a^2 + l^2} - a
    \right]
\end{equation}

Ersetzt man die  Koordinate $a$ durch $x$ und  integriert anschliessend \"uber
die  Gesamtfl\"ache  ausserhalb  des  Drahtes  (also  vom  Radius  $\rho$  bis
unendlich) ergibt dies:

\begin{align}
    \frac{\Phi}{i} 
    = 
    2 \mu \cdot
    \int_{\rho}^{\infty}
    \left[ 
        \frac{\sqrt{x^2+l^2}}{x} - 1
    \right]
    dx
    & =
    2 \mu \cdot
    \left[ 
        \sqrt{x^2 + l^2} - x - l \log{\frac{l + \sqrt{x^2+l^2}}{x}}
    \right]_{\rho}^{\infty}
    \label{eq:sSCW:4}
    \\
    & =
    2 \mu \cdot
    \left[ 
        l \cdot \log{\frac{l+\sqrt{l^2+\rho^2}}{\rho}} - \sqrt{l^2+\rho^2} + \rho
    \right]
    \label{eq:sSCW:5}
    \\
    & \approx
    2 \mu l \left[ 
        \log{\frac{2l}{\rho}} - 1
    \right]
    \label{eq:sSCW:6}
\end{align}

Gem\"ass    Gleichung   \code{TODO}    ist    dies   der    Anteil   an    der
Selbstinduktivit\"at,  welcher vom  magnetischen Feld  ausserhalb des  Leiters
verursacht wird. Zus\"atzlich muss jedoch auch  das Feld im Innern des Leiters
ber\"ucksichtigt werden.

\schematic{selfStraightCylWireCrossSection}

Die   Feldst\"arke  am   Punkt   $Q   (x,0)$  im   Innern   des  Drahtes   ist
$\frac{2ix\mu}{\rho^2}$. Der auf den  Strom normierte Fluss in  einem Element des
Drahtes mit L\"ange $l$ und der Breite $dx$ ist somit

\begin{equation}
    \label{eq:sSCW:7}
    \frac{d\Phi}{i} 
    = 
    \frac{2ilxdx}{\rho^2}
\end{equation}

Integration  dieses  Ausdrucks  von  $0$ bis  $\rho$  ergibt  den  Gesamtfluss
innerhalb des Drahtes:

\begin{equation}
    \label{eq:sSCW:8}
    \frac{\Phi}{i} = \frac{\mu \cdot li}{\rho^1}
    \cdot
    \int_{-1}^{\rho} 2xdx
    =
    li\mu
\end{equation}

Da die  Feldlinien innerhalb  des Leiters nicht  den Gesamten  Querschnitt des
Leiters schneiden m\"ussen sie gewichtet  werden, proportional zu der Fl\"ache
des Leiters, die geschnitten wird:

\begin{equation}
    \label{eq:sSCW:8}
    i \cdot L_2 
    = 
    \mu l \cdot \int_{0}^{\rho} \frac{2ix}{\rho^2} \cdot \frac{x^2}{\rho^2}dx
    = 
    \frac{\mu \cdot li}{2}
\end{equation}

Somit:

\begin{equation}
    \label{eq:sSCW:9}
    L_2 = \frac{\mu l}{2}
\end{equation}

Zusammenf\"ugen der Gleichungen \ref{eq:sSCW:9} und \ref{eq:sSCW:5} respektive
\ref{eq:sSCW:6} ergibt:

\begin{align}
    L 
    & = 
    2 \mu \cdot 
    \left[
        l \cdot \log{\frac{l+\sqrt{l^2+\rho^2}}{\rho}}
        - 
        \sqrt{l^2+\rho^2}
        +
        \frac{l}{4}
        +
        \rho
    \right]
    \label{eq:sSCW:10}
    \\
    & \approx
    2 \mu \cdot 
    \left[
        \log{\frac{2l}{\rho}} - \frac{3}{4}
    \right]
    \label{eq:sSCW:11}
\end{align}

Wenn  das  Magnetfeld  innerhalb  des  Leiters  vernachl\"assigt  werden  kann
(hohe   Frequenzen   \tikz\draw[-{Stealth}](0,0)  --   (0.5,0);   Skin-Effekt,
vernachl\"assigbarer Drahtdurchmesser), vereinfacht sich \ref{eq:sSCW:11} zu:

\begin{equation}
    \label{eq:sSCW:12}
    L_2 
    \approx 
    2 \mu \cdot 
    \left[
        \log{\frac{2l}{\rho}} - 1
    \right]
\end{equation}


% ---------------------------------------------------------------------------- %
\section{Gegenseitige Induktivit\"at von zwei parallelen Dr\"ahten}
\label{sec:mutualStraightWires}
% ---------------------------------------------------------------------------- %

\schematic{mutualStraightWires}

Die   gegenseitige  Induktivit\"at   zweier  Dr\"ahte   im  Abstand   $d$  mit
gegenl\"aufigem Strom $i$ berechnet sich  \"uber den auf $i$ normierten Fluss,
welcher den jeweils anderen Leiter durchfliesst. Dazu integriert man Gleichung
\ref{eq:sSCW:4}  von $d$  bis $\infty$  anstatt von  $\rho$ bis  $\infty$, was
folgenden Ausdruck ergibt:

\begin{align}
    M
    & =
    2 \mu
    \left[
        l \cdot \log{\frac{l+\sqrt{l^2+d^2}}{d}} - \sqrt{l^2 + d^2} + d
    \right]
    \label{eq:mSW:1}
    \\
    & \approx
    2 \mu l
    \left[
        \log{\frac{2l}{d}} - 1 + \frac{d}{l}
    \right]
    \label{eq:mSW:2}
\end{align}

\ref{eq:mSW:1}   und    \ref{eq:mSW:2}   sind   sehr   genau,    solange   der
Leiterquerschnitt  im  Vergleich  zum  Abstand  $d$  vernachl\"assigbar  klein
ist. Ihre Genauigkeit ist ebenfalls  noch zufriedenstellend, solange $l$ gross
im Vergleich zu $d$, selbst wenn  der Leiterquerschnitt gross ist im Vergleich
zu $d$.


% ---------------------------------------------------------------------------- %
\section{Selbstinduktivit\"at eines Stromkreises}
\label{sec:selfReturnCircuit}
% ---------------------------------------------------------------------------- %

Zur  Bestimmung   der  Selbstinduktivit\"at  eines   geschlossenen  Stromkreis
gem\"ass    Abbildung     \ref{fig:mutualStraightWires}    muss     man    die
Selbstinduktivit\"at   jeder    einzelnen   Litze   und    ihre   gegenseitige
Induktivit\"aten miteinander verrechnen:

\begin{equation}
    \label{eq:sRC:1}
    L_{tot} = 2 L_{Litze} - 2 M
\end{equation}

Nimmt  man  die  approximierten  Werte  aus  den  entsprechenden  Gleichungen,
erh\"alt man hierf\"ur:

\begin{equation}
    \label{eq:sRC:2}
    L 
    = 
    4 \mu_0 l 
    \left[ 
        \log{\frac{d}{\rho}} + \frac{\mu_r}{4}
    \right]
\end{equation}

% ---------------------------------------------------------------------------- %
%\section{Gegenseitige Induktivit\"at zweier paralleler Dr\"ahte nach Neumann's Formel}
%\label{sec:neumann
% ---------------------------------------------------------------------------- %

% ---------------------------------------------------------------------------- %
\section{Gegenseitige Induktivit\"at zweier linearer Leiter in einer geraden Linie}
\label{sec:mutualInStraightLine}
% ---------------------------------------------------------------------------- %

\schematic{mutualInStraightLine}

Die    Selbstinduktivit\"at   der    Leitung   $AB$    wurde   in    Abschnitt
\ref{sec:selfStraightCylWire} bestimmt  durch die Integration  des Magnetfelu
\"uber der  Fl\"ache $ABB'A$,  wobei $A'$ und  $B'$ ins  Unendliche verschoben
werden.

Analog dazu kann die gegenseitige Induktion  der Leiter $AB$ und $BC$ bestimmt
werden, indem  das vom Strom  in $AB$  verursachte Magnetfeld in  der Fl\"ache
$S_2 = BCC'B'$  integriert wird (wobei $B'$ und $C'$  ebenfalls ins Unendliche
verschoben werden).

Das Magnetfeld im Punkt $P (x,y)$, verursacht durch Strom $i$ in $AB$, lautet:

\begin{equation}
    \label{eq:mISL:1}
    H_{P}
    =
    \frac{i\mu_0}{x}
    \left[
        \frac{y}{\sqrt{x^2+y+2}} - \frac{y - l}{\sqrt{x^2+(y-l)^2}}
    \right]
\end{equation}

Der Gesamte auf $i$ normierte magnetische Fluss in $S_2$ ist somit:

\begin{align}
    \frac{\Phi}{i}
    & =
    \mu_0 
    \int_{0}^{\infty} \frac{dx}{x} \int_{0}^{l+m}
    \left[
        \frac{y dy}{\sqrt{x^2+y^2}} - \frac{(y-l) dy}{\sqrt{x^2+(y-l)^2}}
    \right]
    \nonumber
    \\
    &=
    \mu_0
    \int_{0}^{\infty}
    \left[
        \sqrt{x^2 + (l+m)^2)} - \sqrt{x^2+l^2} - \sqrt{x^2+m^2}+x
    \right]
    \frac{dx}{x}
    \nonumber
    \\
    & = \mu_0
    \Bigg[
        \sqrt{x^2 + (l+m)^2} - \sqrt{x^2+l^2} - \sqrt{x^2+m^2}
        \nonumber
        \\
        & ~~~~~~~~~   + x - l \cdot \log{\frac{l+m+\sqrt{x^2+(l+m)^2}}{l+\sqrt{x^2+l^2}}}
        \nonumber
        \\
        & ~~~~~~~~~   - m \cdot \log{\frac{l+m+\sqrt{x^2+(l+m)^2}}{m+\sqrt{x^2+l^2}}} ~
    \Bigg]_0^{\infty}
    \label{eq:mISL:2}
    \\
    & \approx
    \mu_0 \cdot \left[ l \cdot \log{\frac{l+m}{l}} + m \cdot \log{\frac{l+m}{m}} \right]
    \label{eq:mISL:3}
\end{align}

Die Ann\"aherung \ref{eq:mISL:3}  ist von guter Qualit\"at,  solange $m$ nicht
allzu gross wird und der Radius des Leiters $BC$ klein ist.

% ---------------------------------------------------------------------------- %
%\section{Definition von Selbstinduktivit\"at in einem offenen Kreis}
% ---------------------------------------------------------------------------- %

% ---------------------------------------------------------------------------- %
\section{Selbstinduktivit\"at einer geraden Rechteckstange}
\label{sec:selfRectBar}
% ---------------------------------------------------------------------------- %

Die   Selbstinduktivit\"at  einer   geraden   Rechteckstange  entspricht   der
gegenseitigen Induktivit\"at zweier gerader Filamente gleicher L\"ange, welche
im  Abstand  der  geometrischen  Mitteldistanz  $R$  der  Querschnittsfl\"ache
zueinander liegen.

\schematic{rectangleCrossSection}

Die approximierte  Formel f\"ur die Selbstinduktivit\"at  einer Rechteckstange
ist somit:

\begin{equation}
    L = 2 \mu_0 l \left[ \log{\frac{2l}{R}} - 1 + \frac{R}{l} \right]
    \label{eq:sRB:1}
\end{equation}

Maxwell  gibt die  Formel f\"ur  $R$  f\"ur ein  Rechteck mit  Seiten $a$  und
$b$  im  zweiten Band  von  \emph{A  Treatise  on Electricity  and  Magnetism}
\cite{ref:maxwell:gmd} mit dem folgenden Ausdruck an:

\begin{align}
    \log{R}
    & =
    \log{\sqrt{a^2+b^2}} 
    \nonumber\\
    & - \frac{1}{6} \cdot \frac{a^2}{b^2} \cdot \log{\sqrt{1+\frac{b^2}{a^2}}}
    - \frac{1}{6} \cdot \frac{b^2}{a^2} \cdot \log{\sqrt{1+\frac{a^2}{b^2}}}
    \nonumber\\
    & + \frac{2}{3} \cdot \frac{a}{b}\tan^{-1}{\frac{b}{a}}
    + \frac{2}{3} \cdot \frac{b}{a}\tan^{-1}{\frac{a}{b}}
    - \frac{25}{12}
    \label{eq:sRB:2}
\end{align}

Es kann gezeigt werden, dass im Falle eines Rechtecks die folgende Beziehung stets
in sehr guter N\"aherung f\"ur alle $\alpha$ und $\beta$ zutrifft:

\begin{equation}
    \label{eq:sRB:3}
    R = 0.2235 (\alpha + \beta)
\end{equation}

Tabelle   \ref{tab:meanDist}  fasst   einige  Seitenverh\"altnisse   in  einem
Vergleich mit $R$ und dem Verh\"altnis $\frac{R}{\alpha + \beta}$ zusammen.

\begin{table}
    \centering
    \caption{%
        $\alpha$ und $\beta$  sind die Seiten des Rechtecks,  $R$ die mittlere
        geometrische Distanz seiner Querschnittsfl\"ache.%
    }
    \label{tab:meanDist}
    \begin{tabular}{rclcc}
        \toprule
        $\alpha$ & $\div$ & $\beta$ & $R$ & $\frac{R}{\alpha + \beta}$ \\
        \midrule
        $1$    & $\div$ & $1$     & $0.44705 \alpha$ & $0.22353$ \\
        $1.25$ & $\div$ & $1$  & $0.40235 \alpha$ & $0.22353$ \\
        $1.5$  & $\div$ & $1$   & $0.37258 \alpha$ & $0.22355$ \\
        $2$    & $\div$ & $1$     & $0.33540 \alpha$ & $0.22360$ \\
        $4$    & $\div$ & $1$     & $0.27961 \alpha$ & $0.22369$ \\
        $10$   & $\div$ & $1$    & $0.24596 \alpha$ & $0.22360$ \\
        $20$   & $\div$ & $1$    & $0.23463 \alpha$ & $0.22346$ \\
        $1$    & $\div$ & $0$     & $0.22315 \alpha$ & $0.22315$ \\
        \bottomrule
    \end{tabular}
\end{table}

Es ist somit m\"oglich, die  komplizierte Formel \ref{eq:sRB:2} f\"ur $R$ ohne
grossen Verlust an Pr\"azision  durch die Vereinfachte Formel \label{eq:sRB:1}
in Gleichung \ref{eq:sRB:3} zu ersetzen. Eingesetzt ergibt dies:

\begin{equation}
    \label{eq:sRB:4}
    L 
    = 
    2 \mu_0 l \left[ \log{\frac{2l}{\alpha + \beta}} 
    + \frac{1}{2} 
    + \frac{0.2235(\alpha + \beta)}{l} \right]
\end{equation}


% ---------------------------------------------------------------------------- %
\section{Zwei parallele Rechteckstangen -- Gegenseitige und Eigeninduktivit\"at}
\label{sec:mutualRectBar}
% ---------------------------------------------------------------------------- %

\schematic{rectangleTwoCrossSection}

Die gegenseitige Induktivit\"at zweiter quadratischer oder rechteckiger Leiter
ist  gleich  der  gegenseitigen   Induktivit\"at  zweier  paralleler  Dr\"ahte
gleicher  L\"ange im  Abstand  der mittleren  geometrischen  Distanz der  zwei
Fl\"achen voneinander. Wir nehmen somit  Gleichung \ref{eq:mSW:2} und ersetzen
$d$ durch die m.g.D. $R$ der beiden Rechteckstangen.

\begin{equation}
    M
    =
    2 \mu l
    \left[
        \log{\frac{2l}{R}} - 1 + \frac{R}{l}
    \right]
    \label{eq:mRB:1}
\end{equation}

Im   Falle    von   quadratischen   Querschnitten   entspricht    die   g.m.D.
in    guter    N\"aherung    der    Distanz    der    Zentren    der    beiden
Quadrate. Im  Falle  von  parallelen   Quadraten  (oberer  Fall  in  Abbildung
\ref{fig:rectangleTwoCrossSection}) ist die g.m.D.  ein bisschen gr\"osser als
der Abstand der  Zentren, im Falle von diagonal gewinkelten  Quadraten ist die
g.m.D. ein wenig kleiner als der Abstand der Zentren.


% ---------------------------------------------------------------------------- %
\section{Selbstinduktivit\"at eines Quadrats}
\label{sec:selfSquare}
% ---------------------------------------------------------------------------- %

\schematic{selfSquare}

Die Selbstinduktivit\"at  eines Quadrates  setzt sich  zusammen aus  der Summe
der  Selbstinduktivit\"at  der  einzelnen Seiten  weniger  ihre  gegenseitigen
Induktivit\"aten,  wobei die  gegenseitige Induktivit\"at  zweier rechtwinklig
stehender Seiten verschwindet\footnotemark:

\begin{equation}
    \label{eq:sS:1}
    L = 4 L_1 - 4M
\end{equation}

\footnotetext{%
    Gem\"ass der  Neumann'schen Formel  f\"ur die  gegenseitige Induktivit\"at
    zweier  Str\"ome  $\vec{I_1}$  und   $\vec{I_2}$,  welche  durch  separate
    Stromkreise fliessen:

    \begin{equation}
        \label{eq:sS:2}
        M = \frac{\mu_0}{4\pi} \oint \oint \frac{d\vec{I_1} \cdot d\vec{I_2}}{r}
    \end{equation}

    Das   Skalarprodukt   zweier    normal   zueinander   stehender   Vektoren
    ist nat\"urlich null.
}


Wir ben\"otigen also die Gleichunge \ref{eq:sSCW:10} und \ref{eq:mSW:1}, wobei
$l$, respektive $d$ durch $a$ ersetzt werden. Es folgt:


\begin{align}
    L 
    & = 
    2a \mu 
    \left[ 
        \log{\frac{a + \sqrt{a^2+\rho^2}}{\rho}} - \sqrt{1-\frac{\rho^2}{a^2}} + \frac{1}{4} + \frac{\rho}{a}
    \right]
    \label{eq:sS:3}
    \\
    M 
    & = 
    2a \mu 
    \left[ 
        \log{\frac{a + \sqrt{2a^2}}{a}} - \sqrt{2} + 1
    \right]
    \label{eq:sS:4}
\end{align}

Wenn  die  Seitenl\"ange   $a$  des  Quadrats  gross  ist   im  Vergleich  zum
Drahtdurchmesser  $\rho$, kann  $\frac{\rho^2}{a^2}$ vernachl\"assigt  werden.

\begin{align}
    L_1 - M
    & =
    2a\mu_0 
    \left[
        \log{\frac{2a}{\rho(1+\sqrt{2})}} - 1.75 + \sqrt{2} + \frac{\rho}{a}
    \right]
    \label{eq:sS:5}
    \\
    L 
    &= 
    4(L_1 - M) 
    =
    8a\mu_0
    \left[
        \log{\frac{a}{\rho}}  - \log{\frac{1+\sqrt{2}}{2}} + \frac{\rho}{a} - 0.3358
    \right]
    \label{eq:sS:5}
    \\
    L
    & =
    8a \mu_0
    \left[ 
        \log{\frac{a}{\rho}} + \frac{\rho}{a} - 0.524
    \right]
\end{align}


% ---------------------------------------------------------------------------- %
\section{Selbstinduktivit\"at eines Rechtecks}
\label{sec:selfRect}
% ---------------------------------------------------------------------------- %

Es  werden die  beiden  F\"alle eines  Rechtecks aus  rundem  Draht und  eines
Rechtecks aus rechteckigem Draht betrachtet.

% ---------------------------------------------------------------------------- %
\subsection{Runder Drahtquerschnitt}
\label{subsec:selfRect:round}
% ---------------------------------------------------------------------------- %
\schematic{selfRectRound}

Die Selbstinduktivit\"at eines Rechtecks mit Seitenl\"angen $a$ und $b$ ist:

\begin{equation}
    \label{eq:sRro:1}
    L 
    = 
    2 
    \left[
        L_a + L_b - M_a - M_b
    \right]
\end{equation}

Wobei  $L_a$  und  $L_b$  die Selbstinduktivit\"aten  der  beiden  Seiten  $a$
respektive $b$  sind, und $M_a$  und $M_b$ die  gegenseitigen Induktivit\"aten
der  beiden jeweils  einander gegen\"uberliegenden  Seiten $a$  respektive $b$
(also  linke Seite  $a$ auf  rechte Seite  $a$ und  umgekehrt, daher  noch mit
Faktor $2$ multipliziert; analog f\"ur $b$).

Erneut bedienen wir uns der Formeln \ref{eq:sSCW:10} und \ref{eq:mSW:1}, unter
Vernachl\"assigung der quadratischen Terme. Eingesetzt in \ref{eq:sRro:1}:

\begin{align}
    L 
    & = 
    4 \mu_0
    \left[
        a \log{\frac{2a}{\rho}} - \frac{3}{4}a + \rho
    \right]
    + 4 \mu_0
    \left[
        b \log{\frac{2b}{\rho}} - \frac{3}{4}b + \rho
    \right]
    \nonumber\\
    & -
    4 \mu_0
    \left[
        a\log{\frac{a+\sqrt{a^2+b^2}}{b}} - \sqrt{a^2+b^2} + b
    \right]
    \nonumber\\
    & -
    4 \mu_0
    \left[
        b\log{\frac{b+\sqrt{a^2+b^2}}{a}} - \sqrt{a^2+b^2} + a
    \right]
    \label{eq:sRro:2}
\end{align}

Ersetzt man den  Term $\sqrt{a^2+b^2}$ durch die Diagonale  $d$, l\"asst sich
dies vereinfachen zu:

\begin{equation}
    \label{eq:sRro:3}
    L
    =
    \left[
        (a+b) \log{\frac{2ab}{\rho}} - a\log{(a+d)} - b\log{(b+d)} - \frac{7}{4}(a+b)+2(d+\rho)
    \right]
\end{equation}


% ---------------------------------------------------------------------------- %
\subsection{Rechteckiger Drahtquerschnitt}
\label{subsec:selfRect:rect}
% ---------------------------------------------------------------------------- %
\schematic{selfRect}

Gem\"ass Abbildung  \ref{fig:selfRect} besteht  der Leiter aus  einem Rechteck
mit  \"ausseren  Seitenl\"angen $a$  und  $b$,  und  der Querschnitt  hat  die
Kantenl\"angen $\alpha$ und $\beta$.

Setzt man  $a$ und  $b$ in den  Gleichungen \ref{eq:sRB:4}  und \ref{eq:mSW:1}
f\"ur  die L\"ange  der  Leiter bzw.  deren Abstand  ein,  ergeben sich  f\"ur
die  Selbstinduktivit\"at   und  die  Gegeninduktivit\"at  f\"ur   die  Seiten
$a$\footnotemark:

\footnotetext{Analog werden nat\"urlich $L_b$ und $M_b$ berechnet.}

\begin{align}
    L_a
    & =
    2
    \left[
        a \log{\frac{2a}{\alpha + \beta}} + \frac{a}{2} + 0.2235(\alpha+\beta)
    \right]
    \label{eq:sRre:1}
    \\
    M_a
    & =
    2
    \left[
        a \log{\frac{a+\sqrt{a^2+b^2}}{b}} - \sqrt{a^2+b^2} + b
    \right]
    \label{eq:sRre:2}
\end{align}

Wie  im Falle  des runden  Leiterquerschnitts gilt  $L =  2(L_a+L_b-M_a-M_b)$,
womit die gesamte Induktivit\"at auf folgenden Ausdruck kommt:

\begin{align}
    L
    & =
    4
    \left[
        a \log{\frac{2ab}{(\alpha 
        + \beta)(a + \sqrt{a^2+b^2})} 
        + \frac{a}{2} - b + \sqrt{a^2+b^2} 
        + 0.2235(\alpha + \beta)_a}
    \right]
    \nonumber\\
    & +
    4
    \left[
        b \log{\frac{2ab}{(\alpha 
        + \beta)(b + \sqrt{a^2+b^2})} 
        + \frac{b}{2} - a + \sqrt{a^2+b^2} 
        + 0.2235(\alpha + \beta)_b}
    \right]
    \label{eq:sRre:3}
\end{align}

$(\alpha+\beta)_a$  und $(\alpha+\beta)_b$  sind dabei  die Leiterquerschnitte
entlang  der  Rechtecksseiten  $a$   respektive  $b$. Substituiert  man  $d  =
\sqrt{a^2+b^2}$ und  setzt voraus, dass $(\alpha+\beta)_a  = (\alpha+\beta)_b$
(dass also  der Leiterquerschnitt  \"uber das  gesamte Rechteck  uniform ist),
vereinfacht sich Gleichung \ref{eq:sRre:3} zu:

\begin{equation}
    \label{eq:sRre:4}
    L
    =
    4
    \left[
        (a+b) 
        \log{\frac{2ab}{\alpha+\beta}} 
        - a\log(a+d) 
        - b\log{(b+d)} 
        - \frac{a+b}{2} 
        + 2d 
        + 0.447(\alpha+\beta)
    \right]
\end{equation}

Im Falle eines Quadrats ($a=b$) ergibt sich:

\begin{equation}
    \label{eq:sRre:5}
    L
    =
    8a
    \left[
        \log{\frac{a}{\alpha+\beta}} 
        + 0.2235\frac{\alpha+\beta}{a}+0.726
    \right]
\end{equation}

Im  zus\"atzlichen Falle  eines  quadratischen  Leiterquerschnitts ($\alpha  =
\beta$):

\begin{equation}
    \label{eq:sRre:6}
    L
    =
    8a
    \left[
        \log{\frac{a}{\alpha}} 
        + 0.447\frac{\alpha}{a}+0.33
    \right]
\end{equation}


% ---------------------------------------------------------------------------- %
\section{Gegenseitige Induktivit\"at zweier gleicher parelleler Rechtecke}
\label{sec:mutualTwoEqualRect}
% ---------------------------------------------------------------------------- %

\schematic{mutualTwoEqualRect}

Die  gegenseitige  Induktivit\"at  zweiter gleicher  Rechtecke  $1,2,3,4$  und
$5,6,7,8$  im  konstanten  Abstand  $d$   setzt  sich  aus  den  gegenseitigen
Induktivit\"aten   der   verschiedenen   Seiten   zusammen. Auch   hier   wird
Formel  \ref{eq:mSW:1}  als  Ausgangspunkt genommen. F\"ur  die  verschiedenen
gegenseitigen Induktivit\"aten erh\"alt man:

\begin{align}
    M
    & =
    2 (M_{15} - M_{17}) + 2(M_{26} - M_{28})
    \label{eq:mTER:1}\\
    M_{15}
    & =
    2 
    \left[
        b \log{\frac{b + \sqrt{b^2+d^2}}{d}} - \sqrt{b^2+d^2} + d
    \right]
    \label{eq:mTER:2}\\
    M_{17}
    & =
    2 
    \left[
        b \log{\frac{b + \sqrt{a^2+b^2+d^2}}{\sqrt{a^2+d^2}}} - \sqrt{a^2+b^2+d^2} + \sqrt{a^2+d^2}
    \right]
    \label{eq:mTER:3}\\
    M_{26}
    & =
    2 
    \left[
        a \log{\frac{a + \sqrt{a^2+d^2}}{d}} - \sqrt{a^2+d^2} + d
    \right]
    \label{eq:mTER:4}\\
    M_{28}
    & =
    2 
    \left[
        a \log{\frac{a + \sqrt{a^2+b^2+d^2}}{\sqrt{b^2+d^2}}} - \sqrt{a^2+b^2+d^2} + \sqrt{b^2+d^2}
    \right]
    \label{eq:mTER:5}
\end{align}

Man  beachte, dass  es  hier  um die  gegenseitige  Induktivit\"at der  beiden
Rechteckschlaufen   geht,  und   nicht   um  die   Selbstinduktivit\"at. Daher
interessieren $M_{13}$, $M_{24}$, $M_{57}$ und $M_{68}$ nicht.

F\"ugt man die Terme zusammen, ergibt sich f\"ur $M$:


\begin{align}
    M
    =
    4
    & \Bigg[
        a 
        \log{\left( 
            \frac{a + \sqrt{a^2+d^2}}{a + \sqrt{a^2+b^2+d^2}} \cdot \frac{\sqrt{b^2+d^2}}{d}
        \right)}
    \nonumber\\
    & +
        b 
        \log{\left( 
            \frac{b + \sqrt{b^2+d^2}}{b + \sqrt{a^2+b^2+d^2}} \cdot \frac{\sqrt{a^2+d^2}}{d}
        \right)}
    \Bigg]
    \nonumber\\
    +
    & \bigg[
        \sqrt{a^2+b^2+d^2} - \sqrt{a^2+d^2} - \sqrt{b^2+d^2} + d
    \bigg]
    \label{eq:mTER:6}\\
\end{align}

Im Falle eines Quadrats mit $b=a$ vereinfacht sich dies zu:

\begin{align}
    M 
    & =
    8
    a \log{\left(
        \frac{a + \sqrt{a^2+d^2}}{a + \sqrt{2a^2+d^2}} 
        \cdot
        \frac{\sqrt{a^2+d^2}}{d}
    \right)}
    \nonumber\\
    & +
    8
    \bigg[
        \sqrt{2a^2+d^2} - 2\sqrt{a^2+d^2} + d
    \bigg]
    \label{eq:mTER:7}\\
\end{align}


% ---------------------------------------------------------------------------- %
\section{Eigeninduktivit\"at und Gegenseitige Induktivit\"at d\"unner Streifen}
\label{sec:thinTapes}
% ---------------------------------------------------------------------------- %

\todo[inline]{%
    Dieser Abschnitt  sollte meines Erachtens besser  strukturiert werden. Die
    mathematischen  Exkurse   zum  arithmetischen  und   geometrischen  Mittel
    st\"oren  den Leserfluss. Andererseits  sind sie  f\"ur das  Verst\"andnis
    erforderlich.%
}

\begin{minipage}{0.5\textwidth}
    \schematic{tapeSingle}
\end{minipage}
\begin{minipage}{0.5\textwidth}
    \schematic{tapeEdge}
\end{minipage}

\schematic{tapesApart1}

\begin{minipage}{0.5\textwidth}
    \schematic{tapesApart2}
\end{minipage}
\begin{minipage}{0.5\textwidth}
    \schematic{tapesApart3}
\end{minipage}

Die Selbstinduktivit\"at  eines d\"unnen Streifens  vernachl\"assigbarer Dicke
mit L\"ange  $l$ und  Breite $b$ ist  gleich der  gegenseitigen Induktivit\"at
zweier  Leiter  im  Abstand  der   mittleren  geometrischen  Distanz  $R$  des
Querschnitts  des Streifens  (siehe Abschnitt  \ref{sec:selfRectBar} ab  Seite
\pageref{sec:selfRectBar}). Es  gilt  $\log{R_1}   =  \log{b}  -  \frac{3}{2}$
bzw. $R_1 = 0.22313b$.

Vernachl\"assigt  man  die Breite  des  Streifens  im Verh\"altnis  zu  seiner
L\"ange in Formel  \ref{eq:mSW:2}, und setzt $R_1$ f\"ur $d$  ein, ergibt dies
f\"ur die Situation aus Abbildung \ref{fig:tapeSingle}:

\begin{equation}
    \label{eq:tT:1}
    L = 2l \left[\log{\frac{2l}{R_1}} - 1 \right]
      = 2l \left[\log{\frac{2l}{b}}   + \frac{1}{2} \right]
\end{equation}

Sind zwei Streifen in der gleichen  Ebene Kante an Kante platziert, ohne dabei
leitenden Kontakt  herzustellen, wie in Abbildung  \ref{fig:tapeEdge} gezeigt,
muss  die  mittlere  geometrische  Distanz des  einen  Streifens  zum  anderen
eingesetzt werden ($R_2 = 0.89252b$).

\begin{equation}
    \label{eq:tT:2}
    M = 2l \left[ \log{\frac{2l}{R_2}} - 1 \right]
      = 2l \left[ \log{\frac{2l}{b}} - 0.8863 \right]
\end{equation}

F\"ur  einen  geschlossenen  Stromkreis  aus zwei  Streifen  ist  die  gesamte
Selbstinduktivit\"at:

\begin{alignat}{3}
    L &= 2L_1 - 2M  \; &=& \; 4l \log{\frac{R_2}{R_1}} \nonumber\\
      &= 4l \log{4} \; &=& \; 5.545 l
    \label{eq:tT:3}
\end{alignat}

Wenn  die  beiden  Streifen  nicht  in  der  gleichen  Ebene  liegen,  sondern
in  zwei   verschiedenen  zueinander  parallelen  Ebenen,   wie  in  Abbildung
\ref{fig:tapesApart2}  dargestellt, berechnet  sich die  geometrische mittlere
Distanz $R_2$ zwischen den beiden Streifen gem\"ass:

\begin{equation}
    \label{eq:tT:4}
    \log{R_2} 
    = \frac{d^2}{b^2} \log{d} + \frac{1}{2} \left(1-\frac{d^2}{b^2}\right)
    \log{(b^2+d^2)} + 2\frac{d}{b} \tan^{-1}{\frac{b}{d}} - \frac{3}{2}
\end{equation}

Im  vereinfachten Fall  von Abbildung  \ref{fig:tapesApart3} mit  $d=b$ ergibt
sich f\"ur Formel \ref{eq:tT:4}:


\begin{equation}
    \label{eq:tT:5}
    \log{R_2} = \log{b} + \frac{\pi}{2} - \frac{3}{2}
\end{equation}

Mit der nach wie vor g\"ultigen  Beziehung $\log{R_1} = \log{b} - \frac{3}{2}$
f\"ur den Fall eines einzelnen Streifens erh\"alt man:

\begin{equation}
    \label{eq:tT:6}
    \log{\frac{R_2}{R_1}} = \frac{\pi}{2}
\end{equation}

Womit sich f\"ur die gesamte Induktivit\"at ergibt:

\begin{equation}
    \label{eq:tT:6}
    L = 2L_1 - 2M = 4l \log{\frac{R_2}{R_1}} = 4l\frac{\pi}{2} = 2\pi l
\end{equation}


% ---------------------------------------------------------------------------- %
%\section{Anwendung: Shunts mit niedriger Induktivit\"at}
% ---------------------------------------------------------------------------- %

% ---------------------------------------------------------------------------- %
\subsection{Die mittlere geometrische Distanz eines Streifens}
\label{subsec:gmd}
% ---------------------------------------------------------------------------- %

Im  Folgenden  soll  eine  genauere  Formel  f\"ur  die  Induktivit\"at  eines
Streifens  hergeleitet werden,  unter  Benutzung  der mittleren  geometrischen
Distanz und der mittleren arithmetischen Distanz.

Zur Bestimmung der Selbstinduktivit\"at eines d\"unnen, geraden Streifens kann
man  sich  der  Tatsache  behelfen,  dass  die  gesamte  Induktivit\"at  einer
Leiterkonfiguration gleich  der Summe  aller einzelnen  Induktivit\"aten ihrer
Komponenten ist  (die Summe  der Induktivit\"aten  aller Komponenten  auf sich
selbst und auf alle anderen Komponenten).

\schematic{gmd}

Zerlegt man den  Streifen in $n$ Teilstreifen, wie  in Abbildung \ref{fig:gmd}
dargestellt,  tr\"agt jedes  Element  $\frac{1}{n}$ des  gesamten Stromes  und
es  ergeben  sich $n^2$  Induktivit\"aten  insgesamt. Diese  m\"ussen nun  auf
vern\"unftige Art miteinander verrechnet werden.

Die exakte Formel \ref{eq:mSW:1} f\"ur die gegenseitige  Induktivit\"at zweier
Leiter kann umgeschrieben werden als

\begin{equation}
    M
    =
    2 \mu
    \left[
        l \cdot \log{l+\sqrt{l^2+d^2}} - l\log{d} - \sqrt{l^2 + d^2} + d
    \right]
    \label{eq:gmd:1}
\end{equation}

\todo[noline]{Noch unklar, weshalb dies die Mittelwerte sind}

Es  stellt sich  heraus, dass  diese  Gleichung zur  Bestimmung des  gesuchten
Ausdrucks gemittelt  werden muss. Der  Mittelwert von $\log{d}$  resultiert in
der bereits angetroffenen mittleren  \emph{geometrischen} Distanz, wogegen die
Mittelung von  $d$ und  $d^2$ \"uber  die \emph{arithmetische}  respektive die
\emph{quadratische arithmetische} Mittelung bestimmt wird.

Der Durchschnitt von $\log{d}$ f\"ur $n$ Teilstreifen berechnet sich gem\"ass:

\begin{align}
    \frac{1}{n}
    \bigg[
        \log{d_1} + \log{d_2} + \log{d+3} + ... + \log{d_n}
    \bigg]
    &=
    \frac{1}{n}
    \log{\bigg(
        d_1 \cdot d_2 \cdot d_3 \cdot ... \cdot d_n 
    \bigg)}
    \nonumber\\
    &=
    \log{\sqrt[\leftroot{2}n]{d_1 d_2 d_3 ... d_n}}
    \nonumber\\
    &=
    \log{R}
    \label{eq:gmd:2}
\end{align}


% ---------------------------------------------------------------------------- %
\subsection{Die mittlere arithmetische Distanz eines Streifens}
\label{subsec:amd}
% ---------------------------------------------------------------------------- %

\schematic{amd}

Zuerst  bestimmen wir  die arithmetische  mittlere Distanz  $S_1$ des  Punktes
$P$  von  allen   anderen  Punkten  der  Linie   $AB$. Zuerst  betrachten  wir
das  Liniensegment  links  von  $P$. Die mittlere  arithmetische  Distanz  von
$P$  zu   allen  Punkten  dieses  Segments   ist  $\frac{c}{2}$. Die  mittlere
arithmetische Distanz  von $P$ zu  allen Punkten  im rechten Segment  $PB$ ist
$\frac{b-c}{2}$. F\"ur die gesamte Linie $AB$ ergibt sich somit:

\todo[noline]{%
    Woher  diese  Gleichung  pl\"otzlich  kommt,   ist  mir  noch  nicht  ganz
    klar: Begr\"undung suchen.%
}

\begin{align}
    b \cdot S_1
    &=
    \frac{b-c}{2}(b-c) + \frac{c}{2}c = \frac{(b-c)^2}{2}+\frac{c^2}{2}
    \nonumber\\
    S_1
    &=
    \frac{b}{2} - c + \frac{c^2}{2}
    \label{eq:amd:1}
\end{align}

Zur Bestimmung der mittleren arithmetischen Distanz aller Punkte des Streifens
vom Streifen bzw. der mittleren  arithmetischen Distanz des Streifens von sich
selbst  muss $S_1$  \"uber  den Streifen  integriert werden. Substituiert  man
$c=x$:

\begin{align}
    bS_2 
    =
    \int_0^b
    \left(
        \frac{b}{2} - x + \frac{x^2}{b}
    \right)dx
    =
    \left[\frac{bx}{2}-\frac{x^2}{2} + \frac{x^3}{3b}\right]_0^b
    &=
    \frac{b^2}{3}
    \nonumber\\
    S_2
    &=
    \frac{b}{3}
    \label{eq:amd:2}
\end{align}

Als   n\"achstes   wollen   wir  die   mittlere   quadratische   arithmetische
Distanz\footnotemark  $S_1^2$  des  Punktes $P$  vom  Streifen  bestimmen. Man
integriere wie folgt:

\begin{align}
    bS_1^2
    =
    \int_0^b(x-c)^2dx
    &=
    \frac{b^3}{3}-cb^2+c^2b
    \nonumber\\
    S_1^2 = \frac{b^2}{3} - cb + c^2
    \label{eq:amd:3}
\end{align}

\footnotetext{%
    \emph{Beachte}: Die  mittlere quadratische  arithmetische Distanz  $S_1^2$
    ist  nicht  gleich  dem   Quadrat  der  mittleren  arithmetischen  Distanz
    $S_1$. Die $2$ ist kein Exponent in diesem Fall.
}


F\"ur $c=0$,  also die  m.a.q.D eines Endpunktes  des Streifens  vom Streifen,
ergibt sich $S_1^2 = \frac{b}{3}$.

Um die m.a.q.D des Streifens von sich selbst zu finden, integrieren wir analog
zu vorher:

\begin{align}
    bS_2^2
    =
    \int_0^b\left(\frac{b^2}{3}-bx+x^2\right)dx
    &=
    \frac{b^3}{6}
    \nonumber\\
    S_2^2
    &=
    \frac{b^2}{6}
    \label{eq:amd:4}
\end{align}

Nun setzen wir in Formel \ref{eq:gmd:1} die erhaltenen Werte f\"ur $S_1^2$ und
$S_2^2$ ein. Zus\"atzlich ber\"ucksichtigen wir, dass $l$ viel kleiner als $d$
ist.

\begin{align}
    L
    &=
    2
    \left[
        l \log{\left(2l\left[1+\frac{d^2}{4l^2}\right]\right)}
        - l \log{d} - l - \frac{d^2}{2l} + d
    \right]
    \nonumber\\
    &=
    2l
    \left[
        \log{2l} - \log{d} - 1 - \frac{d^2}{4l^2} + \frac{d}{l}
    \right]
    \label{eq:amd:5}\\
    \text{Substituiere:}
    \nonumber\\
    \log{d} &= \log{R_1} = \log{b} - \frac{3}{2}
    \nonumber\\
    d^2 &= S_2^2 = \frac{b^2}{6}
    \nonumber\\
    d   &= S_1^2 = \frac{b}{3}
    \nonumber
\end{align}

Dies ergibt eine genauere Formel f\"ur die Selbstinduktivit\"at eines d\"unnen
Streifens als Formel \ref{eq:tT:6}.

\begin{equation}
    \label{eq:amd:7}
    L = 2l \left[\log{\frac{2l}{b}} + \frac{1}{2} + \frac{b}{3l} - \frac{b^2}{24l^2}\right]
\end{equation}


% ---------------------------------------------------------------------------- %
\section{Selbstinduktivit\"at eines kreisf\"ormig gewickelten Streifens}
\label{sec:selfCircleStrip}
% ---------------------------------------------------------------------------- %

Die Konzepte der mittleren geometrischen und arithmetischen Distanz sollen nun
dazu benutzt werden, die  Selbstinduktivit\"at eines kreisf\"ormig gewickelten
Streifens gem\"ass Abbildung \ref{fig:circleStrip} zu bestimmen.

\schematic{circleStrip}

F\"ur einen kurzen zylindrischen Strom in einem flachen Streifen gibt Rayleigh
folgende Formel, welche genau genug f\"ur die meisten F\"alle ist, solange $b$
nicht zu gross wird im Verh\"altnis zum Zylinderradius $a$:

\begin{equation}
    \label{eq:sCS:1}
    L = 4\pi a 
    \left[
        \log{\frac{8a}{b}} - \frac{1}{2} 
        + \frac{b^2}{32a^2}\left(\log{\frac{8a}{b}} 
        + \frac{1}{4}\right)
    \right]
\end{equation}

Wir werden  diese Formel  als \"Uberpr\"ufung zur  Herleitung der  Methode via
geometrische und arithmetische mittlere Distanzen benutzen.

Die  gegenseitige  Induktivit\"at  zweier paralleler,  konzentrischer  Kreiese
(ganz rechts in Abbildung  \ref{fig:circleStrip}) unter Vernachl\"assigung von
Termen vierter und h\"oherer Ordnung lautet:


\begin{equation}
    \label{eq:sCS:2}
    M 
    = 
    4\pi a
    \left[
        \left(1 + \frac{3d^2}{16a^2}\right)
        \log{\frac{8a}{d}}
        -
        \left(2 + \frac{d^2}{16a^3}\right)
    \right]
\end{equation}

Dies kann man umformen zu:

\begin{equation}
    \label{eq:sCS:3}
    M 
    = 
    4\pi a
    \left[
        \left(1 + \frac{3d^2}{16a^2}\right)
        \log{8a}
        -
        \log{d}
        -
        \frac{3d^2}{16a^2} \log{d}
        -
        2 
        + 
        \frac{d^2}{16a^3}
    \right]
\end{equation}

Um diese Formel auf den Fall eines kurzen Zylinders anwenden zu k\"onnen, muss
im ersten und  letzten Term das quadratische arithmetische  Mittel, im zweiten
Term das geometrische Mittel und im dritten Term das Produkt des quadratischen
arithmetischen Mittels mit dem geometrischen  Mittel bestimmt werden (ein Term
der Form $S_2^2 \cdot \log{R_2}$).

Zur Bestimmung dieses Produkts integriert man wie folgt:

\begin{gather}
    bS_1^2 \log{R_1}
    =
    \int_0^b (x-c)^2 \log(x-c) dx
    =
    \nonumber\\
    \frac{(b-c)^3}{3}
    \left[
        \log(b-c) - \frac{1}{3}
    \right]
    +
    \frac{c^3}{3}
    \left[
        \log{c} - \frac{1}{3}
    \right]
    \label{eq:sCS:4}
\end{gather}

\begin{align}
    b^2 S_2^2\log{R_2}
    &=
    \frac{1}{3}
    \int_0^b
    (b-x)^3\log(b-x)dx 
    + 
    \frac{1}{3}
    \int_0^b
    x^3\log{x}dx
    \nonumber\\
    &-
    \frac{1}{9}
    \int_0^b
    (b-x)^3dx
    -
    \int_0^b
    x^2dx
    \nonumber\\
    &=
    \frac{b^4}{6}
    \left(
        \log{b} - \frac{7}{12}
    \right)
    \label{eq:sCS:5}
\end{align}

\begin{equation}
    \label{eq:sCS:6}
    S_2^2 \log{R_2} = \frac{b^2}{6}\left(\log{b}-\frac{7}{12}\right)
\end{equation}

Nun substituieren wir in Formel \ref{eq:sCS:2} wie folgt:

\begin{align*}
    \log{d} 
    &=
    \log{b} - \frac{3}{2}
    \\
    3d^2 
    &= 
    \frac{b^2}{2}
    \\
    3d^2\log{d}
    &=
    \frac{b^2}{2}\left(\log{b}-\frac{7}{12}\right)
\end{align*}

Dies ergibt:

\begin{align}
    L
    &=
    4\pi a
    \left[
        \left(
            1+\frac{b^2}{32a^2}
        \right)
        \log{8a}
        -
        \log{b}
        +
        \frac{3}{2}
        -
        \frac{b^2}{32a^2}
        \left(
            \log{b}-\frac{7}{12}
        \right)
        -
        2
        -
        \frac{b^2}{96a^3}
    \right]
    \nonumber\\
    &=
    4\pi a
    \left[
        \left(
            1+\frac{b^2}{32a^2}
        \right)
        \log{\frac{8a}{b}}
        -
        \frac{1}{2}
        +
        \frac{b^2}{128a^2}
    \right]
    \label{eq:sCS:7}
\end{align}

was, ein wenig umgeformt, Rayleigh's Formel \ref{eq:sCS:1} entspricht. Der Weg
\"uber  zwei  parallele  Kreise  f\"uhrt via  geometrische  und  arithmetische
Mittelwerte also ebenfalls zum gew\"unschten Ergebnis.


% ---------------------------------------------------------------------------- %
\section{Arithmetische durchschnittliche Distanzen eines Kreises}
\label{sec:amdC}
% ---------------------------------------------------------------------------- %

\todo[inline]{In Appendix verlagern}
\schematic{amdC1}

Die   arithmetische   mittlere   Distanz    eines   Punktes   $P$   (Abbildung
\ref{fig:amdC1} auf einem Kreis vom  Kreis selbst wird berechnet, indem \"uber
den Kreis integriert wird. Mit $PB = 2a \cos{\theta}$ ergibt dies:

\begin{gather}
    \pi a S_1 = \int_0^{\frac{pi}{2}} 2a \cos{\theta} \cdot 2ad\theta = 4a^2
    \nonumber\\
    \S_1 = \frac{4a}{\pi}
    \label{eq:amdC:1}
\end{gather}

Da die  arithmetische mittlere  Distanz f\"ur alle  Punkte des  Kreises gleich
ist, gilt auch $S_2 = \frac{4a}{\pi}$.

F\"ur die quadratische arithmetische mittlere Distanz erh\"alt man:

\begin{gather}
    \pi a S_2^2 = \int_0^{\frac{\pi}{2}} 4a^2 \cos^2{\theta} \cdot 2ad\theta = 2 \pi a^3
    \nonumber\\
    S_1^2 = S_2^2 = 2a^2
    \label{eq:amdC:2}
\end{gather}

F\"ur  einen  Punkt  $P$  ausserhalb oder  innerhalb  des  Kreises  (Abbildung
\ref{fig:amdC2}  ergibt such  aufgrund  von  $\overline{PB}^2 =  a^2  + d^2  +
2ad\cos{\theta}$:

\begin{gather}
    \pi a S_1^2 = a \int_0^{\pi} (a^2+d^2+2ad\cos{\theta})d\theta = \pi a (d^2+a^2)
    \nonumber\\
    S_1^2 = d^2+a^2
    \label{eq:amdC:3}
\end{gather}

\schematic{amdC2}

F\"ur die gesamte Kreisfl\"ache bezogen auf Punkt $P$:

\begin{gather}
    \pi a^2 S_1^2 
    = 
    \int_0^a (d^2+r^2) 2\pi rdr 
    = 
    2\pi\left[\frac{a^2d^2}{2} + \frac{a^4}{4}\right]
    \nonumber\\
    S_1^2 = d^2 + \frac{a^2}{2}
    \label{eq:amdC:4}
\end{gather}

Im Falle von $d=0$ ist $S_1^2 = \frac{a^2}{2}$, die quadratische arithmetische
mittlere Distanz der Fl\"ache des Kreises bezogen auf sein Zentrum.

\schematic{amdC3}

Zur  Bestimmung der  q.a.m.D. einer  Kreisfl\"ache  von sich  selbst kann  man
$\overline{P_1P_2}^2  =  r_1^2 +  r_2^2  -  2r_1r_2\cos{\theta_2 -  \theta_1}$
\"uber die Kreisl\"ache zweimal integrieren. Darauf soll aber an dieser Stelle
verzichtet werden.


% ---------------------------------------------------------------------------- %
\section{Konzentrische Leiter}
\label{sec:concCond}
% ---------------------------------------------------------------------------- %

\schematic{concCond}

Die Selbstinduktivit\"at  eines d\"unnen, geraden  Rohres der L\"ange  $l$ und
Radius $a_2$ ist, wenn $\frac{a_2}{l}$ klein ist,

\begin{equation}
    \label{eq:concCond:1}
    L_2 = 2l \left[\log{\frac{2l}{a_2}} - 1\right]
\end{equation}

Die  gegenseitige  Induktivit\"at eines  solchen  Rohres  auf einen  massiven,
runden  Leiter  innerhalb  des  Rohres  wie  in  Abbildung  \ref{fig:concCond}
dargestellt,  ist  gleich  der  Selbstinduktion des  Rohres,  da  das  gesamte
magnetische Feld des durch das Rohr verursachten Stromes durch den Innenleiter
fliesst.
\todo[noline]{%
    \"Uberlegung  betreffend  der  Feldlinien   noch  nicht  ganz  anschaulich
    verstanden. Klarheit schaffen.%
}
Die Selbstinduktivit\"at eines zylindrischen Innenleiters ist

\begin{equation}
    \label{eq:concCond:2}
    L_1 = 2l \left[\log{\frac{2l}{a_1}} - \frac{3}{4}\right]
\end{equation}

Falls der Strom durch den Innenleiter durch den Aussenleiter zur\"uckfliesst,
ergibt sich f\"ur die Induktivit\"at der Gesamtkonfiguration

\begin{equation}
    \label{eq:concCond:3}
    L = L_1 + L_2 - 2M = L_1 - L_2
\end{equation}

Da $M = L_2$. Somit:

\begin{equation}
    \label{eq:concCond:4}
    L = 2l \left[\log\frac{a_2}{a_1} + \frac{1}{4}\right]
\end{equation}

\schematic{concCond2}

Falls  der \"aussere  Leiter einen  nicht vernachl\"assigbare  Dicke $a_3-a_2$
aufweist   (Abbildung  \ref{fig:concCond2}),   muss  $\log{a}$   in  Gleichung
\ref{eq:concCond:4} durch folgenden Ausdruck ersetzt werden:
\todo[noline]{%
    Hier   ist   das   Paper   ein  bisschen   unklar. Nirgends   vorher   ist
    $a_g$   aufgetaucht. Ebenfalls   ist   $\log{a}$   nicht   als   solches
    in     Gleichung     \ref{eq:concCond:4}    vorhanden,     sondern     ein
    Verh\"altnis. Nachrechnen erforderlich.%
}

\begin{equation}
    \label{eq:concCond:5}
    \log{a_g} = \frac{a_3^2 \log{a_3} - a_2^2 \log{a_2}}{a_3^2-a_2^2} - \frac{1}{2}
\end{equation}

Falls der Strom ein Wechselstrom  von gen\"ugend hoher Frequenz ist, verlagern
sich die Str\"ome auf die Aussenseite  des Innenleiters und die Innenseite des
Aussenleiters, und die Induktivit\"at des gesamten Stromkreises wird zu

\begin{equation}
    \label{eq:concCond:6}
    L = 2l \log{\frac{a_3}{a_1}}
\end{equation}


% ---------------------------------------------------------------------------- %
\section{Mehrere Leiter}
\label{sec:multCond}
% ---------------------------------------------------------------------------- %

\schematic{multCond1}

Wird ein Strom in gleichen Teilen auf Leiter der L\"ange $l$ und Radius $\rho$
verteilt, die  sich im  Abstand $d$ befinden  (Abbildung \ref{fig:multCond1}),
ist  die   Selbstinduktivit\"at  des   geteilten  Leiters  gleich   der  Summe
der   separaten   Selbstinduktivit\"aten   plus  zweimal   ihre   Gegenseitige
Induktivit\"at. Falls $\frac{d}{l}$ klein ist, folgt:

\begin{align}
    L 
    &=
    2
    \left\{
        \frac{l}{2}
        \left[
            \log{\frac{2l}{\rho}} - \frac{3}{4}
        \right]
        +
        \frac{l}{2}
        \left[
            \log{\frac{2l}{d}} - 1
        \right]
    \right\}
    \nonumber\\
    &=
    2l
    \left[
        \log{\frac{2l}{(\rho d)^{\frac{1}{2}}}} - \frac{7}{8}
    \right]
    =
    2l
    \left[
        \log{\frac{2l}{(r_g d)^{\frac{1}{2}}}} - 1
    \right]
    \label{eq:multCond:1}
\end{align}

wobei  $r_g$  die  mittlere  geometrische  Distanz des  Leiters  ist:  $r_g  =
0.7788\rho$.\todo[noline]{Quelle f\"ur $r_g$}

\schematic{multCond2}

Im Falle von  drei Leitern, wie in  Abbildung \ref{fig:multCond2} dargestellt,
ist die Selbstinduktivit\"at analog

\begin{equation}
    2l
    \left[
        \log{\frac{2l}{(r_g d)^{\frac{1}{3}}}} - 1
    \right]
    \label{eq:multCond:2}
\end{equation}

Hierbei ist  $(r_g d^2)^{\frac{1}{3}}$  die mittlere geometrische  Distanz des
Mehrfachleiters.

\schematic{multCond3}

Mit  dem  Prinzip der  geometrischen  mittleren  Distanz  kann man  auch  eine
Leiterkonfiguration  mit  beliebig  vielen Leitern  in  einer  kreisf\"ormigen
Anordnung durchrechnen, wie gezeigt in Abbildung \ref{fig:multCond3}.

F\"ur $n$ \"aquidistant  auf den Umfang verteilte Leiter  ist die geometrische
mittlere Distanz

\begin{align}
    \log{R}
    &=
    \frac{n \log{r_1} + n \log\big(r_{12} \cdot r_{13} \cdot ... \cdot r_{1n}\big)}{n^2}
    \nonumber\\
    &=
    \frac{1}{n}
    \bigg(\log{r_1} + \log\big(r_{12} \cdot r_{13} \cdot ... \cdot r_{1n}\big)\bigg)
    \label{eq:multCond:3}
\end{align}

Dabei  ist $r_1$  die geometrische  mittlere Distanz  eines einzelnen  Leiters
($0.7788\rho$, wobei $\rho$  der Radius des Leiters ist) und  $r_{12}$ ist die
Distanz zwischen den Mitten der Leiter $1$ und $2$ usw.

Falls $a$  der Radius des  Kreises ist,  entlang dessen die  Leiter angeordnet
sind, erh\"alt man

\begin{align}
    \log{R}
    &=
    \frac{1}{n} \log\big(r_1 n a^{n-1}\big)
    \nonumber\\
    \text{oder} \qquad R
    &=
    \big(r_1 n a^{n-1}\big)^{\frac{1}{n}}
    \label{eq:multCond:4}
\end{align}


\schematic{multCond4}
\schematic{multCond5}


% ---------------------------------------------------------------------------- %
\section{Selbstinduktivit\"at einer ``nichtinduktiven'' Anordnung Runder Dr\"ahte}
\label{sec:nonIndWires}
% ---------------------------------------------------------------------------- %

\schematic{nonInd}

% ---------------------------------------------------------------------------- %
\section{Der Fall der nichtinduktiven Windung um einen kreisf\"ormigen Zylinder}
\label{sec:nonIndWindings}
% ---------------------------------------------------------------------------- %
