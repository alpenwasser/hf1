% **************************************************************************** %
\chapter{Mathematische Exkurse}
\label{chap:math}
% **************************************************************************** %

\todo[inline]{``Appendices'' im Inhaltsverzeichnis irgendwie herausbringen.}

% ---------------------------------------------------------------------------- %
\section{Die mittlere geometrische Distanz eines Streifens}
\label{sec:gmd}
% ---------------------------------------------------------------------------- %

Zur Bestimmung der Selbstinduktivit\"at eines d\"unnen, geraden Streifens kann
man  sich  der  Tatsache  behelfen,  dass  die  gesamte  Induktivit\"at  einer
Leiterkonfiguration gleich  der Summe  aller einzelnen  Induktivit\"aten ihrer
Komponenten ist  (die Summe  der Induktivit\"aten  aller Komponenten  auf sich
selbst und auf alle anderen Komponenten).

\schematic{gmd}

Zerlegt man den  Streifen in $n$ Teilstreifen, wie  in Abbildung \ref{fig:gmd}
dargestellt,  tr\"agt jedes  Element  $\frac{1}{n}$ des  gesamten Stromes  und
es  ergeben  sich $n^2$  Induktivit\"aten  insgesamt. Diese  m\"ussen nun  auf
vern\"unftige Art miteinander verrechnet werden.

Die exakte Formel \ref{eq:mSW:1} f\"ur die gegenseitige  Induktivit\"at zweier
Leiter kann umgeschrieben werden als

\begin{equation}
    M
    =
    2 \mu
    \left[
        l \cdot \log{l+\sqrt{l^2+d^2}} - l\log{d} - \sqrt{l^2 + d^2} + d
    \right]
    \label{eq:gmd:1}
\end{equation}

\todo[noline]{Noch unklar, weshalb dies die Mittelwerte sind}

Es  stellt sich  heraus, dass  diese  Gleichung zur  Bestimmung des  gesuchten
Ausdrucks gemittelt  werden muss. Der  Mittelwert von $\log{d}$  resultiert in
der bereits angetroffenen mittleren  \emph{geometrischen} Distanz, wogegen die
Mittelung von  $d$ und  $d^2$ \"uber  die \emph{arithmetische}  respektive die
\emph{quadratische arithmetische} Mittelung bestimmt wird.

Der Durchschnitt von $\log{d}$ f\"ur $n$ Teilstreifen berechnet sich gem\"ass:

\begin{align}
    \frac{1}{n}
    \bigg[
        \log{d_1} + \log{d_2} + \log{d+3} + \cdots + \log{d_n}
    \bigg]
    &=
    \frac{1}{n}
    \log{\bigg(
        d_1 \cdot d_2 \cdot d_3 \cdot \cdots \cdot d_n 
    \bigg)}
    \nonumber\\
    &=
    \log{\sqrt[\leftroot{2}n]{d_1 d_2 d_3 \cdots d_n}}
    \nonumber\\
    &=
    \log{R}
    \label{eq:gmd:2}
\end{align}


% ---------------------------------------------------------------------------- %
\section{Die mittlere arithmetische Distanz eines Streifens}
\label{sec:amd}
% ---------------------------------------------------------------------------- %

\schematic{amd}

Zuerst  bestimmen wir  die arithmetische  mittlere Distanz  $S_1$ des  Punktes
$P$  von  allen   anderen  Punkten  der  Linie   $AB$. Zuerst  betrachten  wir
das  Liniensegment  links  von  $P$. Die mittlere  arithmetische  Distanz  von
$P$  zu   allen  Punkten  dieses  Segments   ist  $\frac{c}{2}$. Die  mittlere
arithmetische Distanz  von $P$ zu  allen Punkten  im rechten Segment  $PB$ ist
$\frac{b-c}{2}$. F\"ur die gesamte Linie $AB$ ergibt sich somit:

\todo[noline]{%
    Woher  diese  Gleichung  pl\"otzlich  kommt,   ist  mir  noch  nicht  ganz
    klar: Begr\"undung suchen.%
}

\begin{align}
    b \cdot S_1
    &=
    \frac{b-c}{2}(b-c) + \frac{c}{2}c = \frac{(b-c)^2}{2}+\frac{c^2}{2}
    \nonumber\\
    S_1
    &=
    \frac{b}{2} - c + \frac{c^2}{2}
    \label{eq:amd:1}
\end{align}

Zur Bestimmung der mittleren arithmetischen Distanz aller Punkte des Streifens
vom Streifen bzw. der mittleren  arithmetischen Distanz des Streifens von sich
selbst  muss $S_1$  \"uber  den Streifen  integriert werden. Substituiert  man
$c=x$:

\begin{align}
    bS_2 
    =
    \int_0^b
    \left(
        \frac{b}{2} - x + \frac{x^2}{b}
    \right)dx
    =
    \left[\frac{bx}{2}-\frac{x^2}{2} + \frac{x^3}{3b}\right]_0^b
    &=
    \frac{b^2}{3}
    \nonumber\\
    S_2
    &=
    \frac{b}{3}
    \label{eq:amd:2}
\end{align}

%Als   n\"achstes   wollen   wir  die   mittlere   quadratische   arithmetische
%Distanz\footnotemark  $S_1^2$  des  Punktes $P$  vom  Streifen  bestimmen. Man
%integriere wie folgt:
Als   n\"achstes   wollen   wir  die   mittlere   quadratische   arithmetische
Distanz$S_1^2$  des Punktes  $P$  vom Streifen  bestimmen. Man integriere  wie
folgt:

\begin{align}
    bS_1^2
    =
    \int_0^b(x-c)^2dx
    &=
    \frac{b^3}{3}-cb^2+c^2b
    \nonumber\\
    S_1^2 = \frac{b^2}{3} - cb + c^2
    \label{eq:amd:3}
\end{align}

%\footnotetext{%
%    \emph{Beachte}: Die  mittlere quadratische  arithmetische Distanz  $S_1^2$
%    ist  nicht  gleich  dem   Quadrat  der  mittleren  arithmetischen  Distanz
%    $S_1$. Die $2$ ist kein Exponent in diesem Fall.
%}


F\"ur $c=0$,  also die  m.a.q.D eines Endpunktes  des Streifens  vom Streifen,
ergibt sich $S_1^2 = \frac{b}{3}$.

Um die m.a.q.D des Streifens von sich selbst zu finden, integrieren wir analog
zu vorher:

\begin{align}
    bS_2^2
    =
    \int_0^b\left(\frac{b^2}{3}-bx+x^2\right)dx
    &=
    \frac{b^3}{6}
    \nonumber\\
    S_2^2
    &=
    \frac{b^2}{6}
    \label{eq:amd:4}
\end{align}



% ---------------------------------------------------------------------------- %
\section{Arithmetische durchschnittliche Distanzen eines Kreises}
\label{sec:amdC}
% ---------------------------------------------------------------------------- %

\todo[inline]{Referenzen in Haupttext}
\schematic{amdC1}

Die   arithmetische   mittlere   Distanz    eines   Punktes   $P$   (Abbildung
\ref{fig:amdC1} auf einem Kreis vom  Kreis selbst wird berechnet, indem \"uber
den Kreis integriert wird. Mit $PB = 2a \cos{\theta}$ ergibt dies:

\begin{gather}
    \pi a S_1 = \int_0^{\frac{pi}{2}} 2a \cos{\theta} \cdot 2ad\theta = 4a^2
    \nonumber\\
    \S_1 = \frac{4a}{\pi}
    \label{eq:amdC:1}
\end{gather}

Da die  arithmetische mittlere  Distanz f\"ur alle  Punkte des  Kreises gleich
ist, gilt auch $S_2 = \frac{4a}{\pi}$.

F\"ur die quadratische arithmetische mittlere Distanz erh\"alt man:

\begin{gather}
    \pi a S_2^2 = \int_0^{\frac{\pi}{2}} 4a^2 \cos^2{\theta} \cdot 2ad\theta = 2 \pi a^3
    \nonumber\\
    S_1^2 = S_2^2 = 2a^2
    \label{eq:amdC:2}
\end{gather}

F\"ur  einen  Punkt  $P$  ausserhalb oder  innerhalb  des  Kreises  (Abbildung
\ref{fig:amdC2}  ergibt such  aufgrund  von  $\overline{PB}^2 =  a^2  + d^2  +
2ad\cos{\theta}$:

\begin{gather}
    \pi a S_1^2 = a \int_0^{\pi} (a^2+d^2+2ad\cos{\theta})d\theta = \pi a (d^2+a^2)
    \nonumber\\
    S_1^2 = d^2+a^2
    \label{eq:amdC:3}
\end{gather}

\schematic{amdC2}

F\"ur die gesamte Kreisfl\"ache bezogen auf Punkt $P$:

\begin{gather}
    \pi a^2 S_1^2 
    = 
    \int_0^a (d^2+r^2) 2\pi rdr 
    = 
    2\pi\left[\frac{a^2d^2}{2} + \frac{a^4}{4}\right]
    \nonumber\\
    S_1^2 = d^2 + \frac{a^2}{2}
    \label{eq:amdC:4}
\end{gather}

Im Falle von $d=0$ ist $S_1^2 = \frac{a^2}{2}$, die quadratische arithmetische
mittlere Distanz der Fl\"ache des Kreises bezogen auf sein Zentrum.

\schematic{amdC3}

Zur  Bestimmung der  q.a.m.D. einer  Kreisfl\"ache  von sich  selbst kann  man
$\overline{P_1P_2}^2  =  r_1^2 +  r_2^2  -  2r_1r_2\cos{\theta_2 -  \theta_1}$
\"uber die Kreisl\"ache zweimal integrieren. Darauf soll aber an dieser Stelle
verzichtet werden.


